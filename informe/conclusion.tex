\par Al emplear embedding de Competencia en Estudiante y a la vez de éste en Escuela, agrupando los datos 
por Mundial, logramos queries eficientes que requieren un único acceso a tabla, con cierta pero no excesiva 
redundancia\footnote{Notar que si se quisiera priorizar la eficiencia de memoria, 
Competencia se podría transformar en una entidad débil y se podrían seguir llevando
a cabo las queries pautadas.}.

\par Los experimentos con sharding nos permitieron entender mejor como funciona esta técnica.
Por ejemplo, creíamos que los documentos se dividirían más o menos de forma igualitaria entre los shards, pero eso no sucede.
Además, esto nos permitió investigar la documentación de RethinkDB y buscar como está implementado el mecanismo de sharding.
RethinkDB tiene funciones especiales que permiten, de manera manual, rebalancear los shards.

\par Para ciertas queries\footnote{En particular, las dos últimas.}, el método Map-Reduce nos resultó ideal.
