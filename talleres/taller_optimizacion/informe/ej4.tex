
\subsection{Ejercicio 4}
%4:
%En el b) no puede usar el indice ya que esta sobre firstname y no rtrim(ltrim(firstname)).

En este ejercicio tenemos sólo un índice. Éste es unclustered y es sobre la tabla member, sobre la columna firstname.

\subsubsection{Queries}

\begin{enumerate}[label=(\alph*)]
\item SELECT * FROM member WHERE firstname = ’UVI
\item SELECT * FROM member WHERE rtrim(ltrim(firstname)) = ’UVI’
\end{enumerate}

\subsubsection{Índice clustered sobre member}
\begin{enumerate}[label=(\alph*)]
\item Realiza un inner join (filas estimadas: 1) sobre el resultado de index seek (filas estimadas: 1) y un RID Lookup (filas estimadas: 1).
\item Realiza un table scsan (filas estimadas: 1000) sobre member.
\end{enumerate}

\subsubsection{Discusión}
\par Se observa el uso del índice (en particular, se realiza un Index Seek) en la primera query pero no en la segunda.
El índice no soporta la operación de búsqueda necesaria para ésta (búsqueda por igualdad sobre rtrim(ltrim(firstname)) en vez de sólo firstname), por lo que se debe emplear un Table Scan.
