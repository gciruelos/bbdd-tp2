Este taller nos permitió aprender varios aspectos distintos de las bases de datos:

\begin{itemize}
\item Cómo obtener el plan de ejecución de una cierta query. Además, como leer e interpretar toda esa información, y entender porqué el plan es ese y no otro.

\item Además, como por ejemplo pasó en el ejercicio 1, viendo el plan de ejecución de una misma query, pero utilizando distintos índices, podemos intentar entender como funciona por dentro el motor de base de datos que estamos usando. Por ejemplo, como implementa los índices, las queries o las operaciones.

\item Por último, como elegir índices para una tabla SQL, teniendo en cuenta las queries que vamos a realizar sobre ella. Esta es una decisión completamente no trivial, porque un índice es una estructura muy cara de mantener, y no es obvio que traiga mejoras inmediatas en la performance de las queries, a menos que se las elija con mucho cuidado.
\end{itemize}
