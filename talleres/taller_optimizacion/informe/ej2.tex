\subsection{Ejercicio 2}
En este ejercicio tenemos 2 índices sobre la misma tabla, uno sobre la columna corp\_no y otro sobre la columna region\_no.

\subsubsection{Queries}

\begin{enumerate}[label=(\alph*)]
\item SELECT * FROM member WHERE region\_no = 9 AND corp\_no = 126
\item SELECT * FROM member WHERE region\_no = 9 AND corp\_no = 368
\end{enumerate}

\subsubsection{Índice sobre corp\_no}
\begin{enumerate}[label=(\alph*)]
  \item Realiza un inner join entre un index seek y un RID lookup (filas estimadas del join y el index seek: 1.5, filas estimadas del RID lookup: 1). Luego filtrará los resultados del join (filas estimadas: 1).
  \item Realiza un inner join entre un index seek y un RID lookup (filas estimadas del join y el index seek: 12, filas estimadsa del RID lookup: 1). Luego filtrará los resultados del join (filas estimadas: 1.3308).
\end{enumerate}

\subsubsection{Índice sobre region\_no}
\begin{enumerate}[label=(\alph*)]
  \item Realiza un table scan (filas estimadas: 1).
  \item Realiza un table scan (filas estimadas: 1.3308).
\end{enumerate}

\subsubsection{Discusión}
%2:
%2 con 126 y 12 con 368
%1109 con region_no = 9
%10000 totales
%a) y b) son iguales con ambos indices (aunque cambian los costos con el primer indice).

En la tabla hay 1109 elementos con region\_no = 9. Además, hay 2 con corp\_no = 126 y 12 con corp\_no = 368. Hay 10000 filas en total.

Notemos, sabiendo cuántos datos hay, que cuando el índice es sobre region\_no, como hay 1009 elementos que matchean, el planificador del motor evalúa que no le conviene recorrer el índice, y directamente recorre toda la tabla.

En el otro caso, como hay muy pocos elementos que matchean el corp\_no de la query, el motor evalúa que es muy conveniente usar el índice, y por esa razón lo hace.
