
\subsection{Ejercicio 4}
%4:
%En el b) no puede usar el indice ya que esta sobre firstname y no rtrim(ltrim(firstname)).
%En el c) tambien, incluso estimando 630 de 10K.

En este ejercicio tenemos sólo un índice. Este índice es unclustered y es sobre la tabla member, sobre la columna firstname.

\subsubsection{Queries}

\begin{enumerate}[label=(\alph*)]
\item SELECT * FROM member WHERE firstname = ’UVI
\item SELECT * FROM member WHERE rtrim(ltrim(firstname)) = ’UVI’
\end{enumerate}

\subsubsection{Índice clustered sobre member}
\begin{enumerate}[label=(\alph*)]
\item Realiza un inner join (filas estimadas: 1) sobre el resultado de index seek (filas estimadas: 1) y un RID Lookup (filas estimadas: 1).
\item Realiza un table scsan (filas estimadas: 1000) sobre member.
\end{enumerate}

\subsubsection{Discusión}

