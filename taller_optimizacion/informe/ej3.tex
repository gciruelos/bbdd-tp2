
\subsection{Ejercicio 3}
%3:
%0 en 99990
%999 en 1000
%10000 en c
%537 en c < 1000
%En el 1)b) se da cuenta que si c.member_no > 99990 entonces m.member_no tambien lo es.
%En 2)a) y b) hace Table Scan, incluso cuando estima solo 11 filas.
%En el c) tambien, incluso estimando 630 de 10K.

En este ejercicio tenemos 2 índices sobre dos tablas distintas, pero ambos sobre la misma columna: member\_no. El primer índice es un índice clustered sobre la tabla member (en este caso, el índice será sobre una clave primaria). El segundo índice es un índice unclustered sobre la tabla charge (o sea, el índice será sobre una clave foránea).

\subsubsection{Queries}

\begin{enumerate}[label=(\alph*)]
\item SELECT c.*, m.* FROM charge c JOIN member m ON c.member\_no = m.member\_no
\item SELECT c.*, m.* FROM charge c JOIN member m ON c.member\_no = m.member\_no WHERE c.charge\_no > 99990
\item SELECT c.*, m.* FROM charge c JOIN member m ON c.member\_no = m.member\_no WHERE m.member\_no < 1000
\end{enumerate}

\subsubsection{Índice clustered sobre member}
\begin{enumerate}[label=(\alph*)]
  \item Realiza un inner join (filas estimadas: 99862.7) de un clustered index scan (filas estimadas: 10000) y un table scan de charge (filas estimadas: 100000).
  \item Realiza un inner join (filas estimadas: 10.9999) de un table scan de charge (filas estimadas: 10.999) y un clustered index seek (filas estimadas: 1).
  \item Realiza un table scan sobre charge (filas estimadas: 632.116) y ordena ese resultado. A eso se le hace un inner join (632.116) con un clustered index seek de member (filas estimadas 999.9).
\end{enumerate}

\subsubsection{Índice unclustered sobre charge}
\begin{enumerate}[label=(\alph*)]
  \item Realiza un inner join (filas estimadas: 99862.7) de un table scan de member (filas estimadas: 10000) y un table scan de charge (filas estimadas: 100000).
  \item Realiza un inner join (filas estimadas: 10.9999) de un table scan de charge (filas estimadas: 10.9999) y un table scan de member (filas estimadas: 10000).
  \item Realiza un inner join (filas estimadas: 632.116) de un table scan de charge (filas estimadas: 632.116) y un table scan de member (filas estimadas: 999.9).
\end{enumerate}

\subsubsection{Discusión}
\par Se puede ver una drástica diferencia entre el índice Clustered y el Unclustered: en el primer caso el motor emplea Index Scan/Seek mientras que en el segundo, Table Scan para todas las queries. 
Si bien esto se debe a que usar un índice Clustered es más eficiente en términos de tiempo de acceso a disco que uno Unclustered,  el motor favorece Table Scan por sobre Index Seek incluso cuando se estiman sólo 11 filas de 100000.
\par Cabe destacar que para el primer índice en la segunda query el motor, al estimar pocas filas, simplemente realiza un Nested Loop Inner Join; en la tercera, al estimar un orden superior de filas, ordena el resultado del Table Scan y lleva a cabo un Merge Join.
